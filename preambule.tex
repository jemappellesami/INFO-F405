%% Packages élémentaires %%
\usepackage[utf8]{inputenc}
\usepackage{mathpazo,etoolbox, graphicx, wrapfig, pbox, fancybox, hyperref, appendix, geometry, amsmath, amssymb, tikz, pgfplots, calc, enumitem, colortbl, listings, pdfpages}
\newcommand{\cst}{\text{c}^{\text{\scriptsize ste}}}
\newcommand{\ddx}{\dfrac{\textrm d ^2 x}{\textrm d t^2}}
\renewcommand{\d}{\mathrm{d}}
\newcommand{\dx}{\mathrm{d}x}
\newcommand{\dy}{\mathrm{d}y}
\newcommand{\dz}{\mathrm{d}z}
\newcommand{\dt}{\mathrm{d}t}
\newcommand{\dxp}{\mathrm{d}x'}
\newcommand{\dyp}{\mathrm{d}y'}
\newcommand{\dzp}{\mathrm{d}z'}
\newcommand{\dtp}{\mathrm{d}t'}
\newcommand{\dvx}{\mathrm{d}\vec x}
\newcommand{\dvxp}{\mathrm{d}\vec x'}
\newcommand{\ch}{\mathrm{ch}}
\newcommand{\sh}{\mathrm{sh}}
\renewcommand{\th}{\mathrm{th}}
\newcommand{\tg}{\mathrm{tg}}
\newcommand{\C}{\textbf{C} }
\newcommand{\Cpp}{\textbf{C}++ }
\newcommand{\ud}[3]{{#1}^{#2} _{\; {#3} }}
\newcommand{\du}[3]{{#1}_{#2} ^{\; {#3} }}
\newcommand{\dd}[3]{{#1}_{#2} _{\; {#3} }}
\newcommand{\uu}[3]{{#1}^{#2} ^{\; {#3} }}
\setlength{\parindent}{0pt}


\geometry{hmargin=2.4cm, vmargin = 2.1cm}
\setlist[itemize]{label=$\bullet$}

%% Couleurs %%
\usepackage{xcolor}
\definecolor{bleu}{RGB}{14, 68, 175}
\definecolor{BGbleu}{RGB}{222, 233, 255 }
\definecolor{BGorange}{RGB}{255, 216, 154}
\definecolor{rouge}{RGB}{201, 0, 0}
\definecolor{vert}{RGB}{14, 137, 0}
\definecolor{BGgris}{RGB}{222,230,230}
\newcommand\rouge[1] {{\color{rouge}{#1}}}
\newcommand\bleu[1] {{\color{bleu}{#1}}}
\newcommand\green[1]{{\color{vert}{#1}}}


%% Polices particulieres
\newcommand\term[1]{\textbf{\bleu{#1}}}

%% Cadres %%
\newcommand\bbm[1]{
\begin{center}
\fcolorbox{black}{BGbleu}{\parbox{\linewidth}{ 
#1
}}
\end{center}}
\newcommand\bo[1]{
\begin{center}
\fcolorbox{black}{BGorange}{\parbox{\textwidth}{ 
#1
}}
\end{center}}

\newcommand\bb[1]{
\begin{center}
\fcolorbox{black}{BGbleu}{\parbox{\textwidth}{ 
\begin{Large}
\begin{center}
#1
\end{center}
\end{Large}
}}
\end{center}}
\renewcommand\bo[1]{
\begin{center}
\fcolorbox{black}{BGorange}{\parbox{\textwidth}{ 
#1
}}
\end{center}}

\newcommand\boite[1]{
\begin{center}
\fbox{\parbox{\textwidth}{#1}}
\end{center}}



\newcommand\aparte[1]{
\begin{center}
\fcolorbox{white}{BGgris}{\parbox{\linewidth}{ \textit{A parte} \\
#1 }}
\end{center}}
\newcommand\bg[2]{
\begin{center}
\fcolorbox{white}{BGgris}{\parbox{\linewidth}{\begin{large} \textit{#1} \end{large} \\

#2 }}
\end{center}}

\newcommand\exemple[1]{
\begin{center}
\fcolorbox{white}{BGgris}{\parbox{\linewidth}{ \textit{Exemple} \\
#1 }}
\end{center}}

%% Commandes %%
\newcommand\imp[1]{\underline{\textbf{#1}}}
\newcommand\eq[1]{\begin{large}
\begin{align*}
#1
\end{align*}
\end{large}}
%% Commandes fantaisistes (cf. Internet) %%
\renewcommand{\parallel}{ \mathbin{\!/\mkern-5mu/\!} }
\newcommand{\q}[1]{{%
\font\larm = larm1000%
\larm%
\char 190}{ \textit{#1} }{%
\font\larm = larm1000%
\larm%
\char 191}}

%% Wrapping %%
\newcommand\wrap[4]{\begin{wrapfigure}[#1]{#2}{#3\textwidth}
#4
\end{wrapfigure}}
%% TikZ
\usetikzlibrary{shapes}
\usetikzlibrary{calc}
\usetikzlibrary{positioning}
\usetikzlibrary{intersections}
\usetikzlibrary{angles}
\usetikzlibrary{quotes}
\newcommand{\drawaxes}[3]{
\coordinate (o) at #1 ;
\draw[->] ($(o) + (-0.1*#2, 0)$) --+ (#2, 0) ;
\draw[->] ($(o) + (0,-0.1*#3)$) --+ (0,#3) ;
}

\newcommand{\drawthickaxes}[5]{
\coordinate (o) at #1 ;
\draw[thick,->] ($(o) + (-0.1*#2, 0)$) --+ (#2, 0) node[anchor = north east]{#4};
\draw[thick,->] ($(o) + (0,-0.1*#3)$) --+ (0,#3) node[anchor = north east]{#5};
}


%% Code %%
\usepackage{listings}
\definecolor{codegreen}{rgb}{0,0.6,0}
\definecolor{codegray}{rgb}{0.5,0.5,0.5}
\definecolor{codepurple}{rgb}{0.58,0,0.82}
\definecolor{backcolour}{RGB}{242,242,242}
\definecolor{codeorange}{RGB}{255,140,0}
% https://gist.github.com/nhtranngoc/88b72d9bfb656a3de227eea38ed80627
\definecolor{background}{RGB}{39, 40, 34}
\definecolor{string}{RGB}{230, 219, 116}
\definecolor{comment}{RGB}{117, 113, 94}
\definecolor{normal}{RGB}{248, 248, 242}
\definecolor{identifier}{RGB}{166, 226, 46}

\newcommand{\code}[1]{\texttt{{\color{codepurple}{#1}}}}
\newcommand{\codep}[1]{\texttt{{\color{codegreen}{#1}}}}

\lstdefinestyle{mystyle}{
    backgroundcolor=\color{backcolour},   
    commentstyle=\color{bleu},
    keywordstyle=\color{codeorange},
    numberstyle=\tiny\color{codegray},
    stringstyle=\color{codepurple},
    basicstyle=\ttfamily\footnotesize,
    breakatwhitespace=false,         
    breaklines=true,                 
    captionpos=b,                    
    keepspaces=true,                 
    numbers=left,                    
    numbersep=5pt,                  
    showspaces=false,                
    showstringspaces=false,
    showtabs=false,                  
    tabsize=2
}

\lstdefinestyle{py}{
	language=python,
    backgroundcolor=\color{backcolour},   
    commentstyle=\color{bleu},
    keywordstyle=\color{codeorange},
    numberstyle=\tiny\color{codegray},
    stringstyle=\color{codepurple},
    basicstyle=\ttfamily\footnotesize,
    breakatwhitespace=false,         
    breaklines=true,                 
    captionpos=b,                    
    keepspaces=true,                 
    numbers=left,                    
    numbersep=5pt,                  
    showspaces=false,                
    showstringspaces=false,
    showtabs=false,                  
    tabsize=2
}

\lstset{style=mystyle}
\lstset{language=C}

\lstdefinestyle{cppstyle}{
basicstyle=\footnotesize\sffamily\color{black},
commentstyle=\color{mygray},
frame=single,
numbers=left,
numbersep=5pt,
numberstyle=\tiny\color{mygray},
keywordstyle=\color{mygreen},
showspaces=false,
showstringspaces=false,
stringstyle=\color{myorange},
tabsize=2
}

\usepackage{framed}

\renewenvironment{leftbar}[1][\hsize]
{%
     \def\FrameCommand
     {%
         {\color{black}\vrule width 3pt}%
         \hspace{7pt}%must no space.
        % \fboxsep=\FrameSep\colorbox{yellow}%
     }%
     \MakeFramed{\hsize#1\advance\hsize-\width\FrameRestore}%
}
{\endMakeFramed}
